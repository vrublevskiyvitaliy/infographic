
%----------------------------------------------------------------------------------------
%	PACKAGES AND OTHER DOCUMENT CONFIGURATIONS
%----------------------------------------------------------------------------------------

\documentclass[letterpaper]{twentysecondcv} % a4paper for A4

\awards {
\begin{itemize}
	\item Particiant of 2014 and 2015 Ukraine ACM ICPC.
	\item 2013: 27th All Ukrainian Olympiad in Informatics, Lugansk, Ukraine\\ \textbf{third diploma}.
\end{itemize}
}

\personal {
\begin{itemize}
	\item Team player, purposeful, responsible, sociable, patient, disciplined and fast learner.
\end{itemize}
}

\goal {
\begin{itemize}
	\item Develop and master my technical and soft skills.
	\item Try to make an impact.
    \item Explore world.
\end{itemize}
}
%----------------------------------------------------------------------------------------
%	 PERSONAL INFORMATION
%----------------------------------------------------------------------------------------

% If you don't need one or more of the below, just remove the content leaving the command, e.g. \cvnumberphone{}

\cvname{Vitalii Vrublevskyi} % Your name
\cvjobtitle{Software Engineer} % Job title/career
\cvlinkedin{https://www.linkedin.com/in/vitalii-vrublevskyi}% LinkedIn

\cvnumberphone{+380680550459} % Phone number
\cvmail{vitalii.vrublevskyi@gmail.com} % Email address

\cvsite{github.com/vrublevskiyvitaliy} % Personal website
\cvhaker{https://www.hackerrank.com/vrublevskyi}
\cvleetcode{https://leetcode.com/vitalii-vrublevskyi}
\cvcodeforce{http://codeforces.com/profile/Steel_Rat11}
\cvkaggle{https://www.kaggle.com/steelrat11}

%----------------------------------------------------------------------------------------

\newcommand\skills{
	\smartdiagram[bubble diagram]{
        \textbf{Programming},
        \textbf{Data}\\\textbf{Structures},
        \textbf{Algorithms},
        \textbf{Machine}\\\textbf{learning},
        \textbf{C++},
        \textbf{Python},
        \textbf{JS},
        \textbf{MySQL},
        \textbf{PHP}
    }
    \\
}

\begin{document}

\makeprofile % Print the sidebar

%----------------------------------------------------------------------------------------
%	 EDUCATION
%----------------------------------------------------------------------------------------
\section{Education}

\begin{twenty} % Environment for a list with descriptions
	\twentyitem
    	{Expected \\ June 2019}
        {Master degree in Informatics}
        {Kyiv, Ukraine}
        {Taras Shevchenko National University of Kyiv}
        {Faculty of Computer Science and Cybernetics}
        
	\twentyitem
    	{June 2017}
        {Bachelor degree with Honours in Informatics}
        {Kyiv, Ukraine}
        {Taras Shevchenko National University of Kyiv}
        {Faculty of Computer Science and Cybernetics}
        
	%\twentyitem{<dates>}{<title>}{<organization>}{<location>}{<description>}
\end{twenty}



%----------------------------------------------------------------------------------------
%	 Projects
%----------------------------------------------------------------------------------------


\section{Projects}
\begin{itemize}
    \item \projectItem
        {Implemented structured data extraction from unstructured text \giturl{https://github.com/vrublevskiyvitaliy/Law_Text_Segmentaion}}
        {The main goal of the project is to divide law documents into sections. My team did it by parsing documents and creating structure of the lists, analysed semantic closeness of paragraphs.}

	\item \projectItem
        {Developed a system for Named Entity Recognition}
        {My team chose the CRF method and researched what features could be used, what annotations of named entities get better results and tested the stability of them at Spanish and Dutch language.}

    \item \projectItem
        {Designed system for automatic discrimination between printed and handwritten text in documents}
        {Used Otsu binarization, dilation and connected components to divide text into words and for each word decided class based on textural and structural features.}
	\item \projectItem
        {Parallel programing}
        {Implemented parallel Dijkstra algorithm using MPI, OpenMP. I explored CUDA for building K-d tree. I used university PARCS approach to solve knapsack problem.}
	%\item \projectItem
    %    {lun.ua}
    %    {Lun.Novostroyki - service for choosing apartments at new buildings.PHP, Python, MySQL, JS, Elasticsearch, Angular 2. Provided ideas to improve project architecture, divided tasks into stages and implemented them. }
    %\item \projectItem
    %    {MP5 Project - WeDesign.Live}
    %    {Web based live collaborative platform for designing with slicer software. Developed JavaScript side of designer, architecture for constructive solid geometry (CSG) technique, implemented tree based data structure which decreased required memory and calculation time.}
	\item \projectItem
        {NBA Totalizator based on Naive Bayes.}
        {Simple predictive model of NBA game based on Naive Bayes approach using results of previous games.}
\end{itemize}
\projectItem
    {See my other projects at \giturl{https://github.com/vrublevskiyvitaliy} }
    {}

\section{Experience}

\begin{twenty} % Environment for a list with descriptions
	\twentyitemwithoutbegin
    	{}
        {Software Engineer}
        {Sep 2015 - Present}
        {Lun.ua, Kyiv, Ukraine}
        {Service for choosing apartments at new buildings.
        {PHP, Python, MySQL, JS, Elasticsearch, Angular 2. Provided ideas to improve project architecture, divided tasks into stages, implemented them.
        }
    }

    \twentyitemwithoutbegin
   	    {}
        {Software Engineer (Remote)}
        {Sep 2015 - Dec 2016 }
        {MP5 Project - WeDesign.Live, London, UK}
        {Web based live collaborative platform for designing with slicer software.
        {JavaScript, C++, Python, Computational Geometry, Linear Algebra. Developed JavaScript side of designer, architecture for constructive solid geometry (CSG) technique which decreased calculation time.
        }
        }
\end{twenty}
\section{Publications}
\begin{twenty}
	\twentyitemwithoutend
    	{2017}
        {Constructing a unified algorithmic platform based on Voronoi diagram.}
        {}
        {PDMU-2017 XXIX International Conference}
        {}
        \end{twenty}
\begin{twenty}
	\twentyitemwithoutend
    	{2017}
        {Greedy approach for solving Art Gallery Problem}
        {}
        {XV International conference "Shevchenkivska Spring 2017"}
        {}
\end{twenty}

\end{document} 
