
%----------------------------------------------------------------------------------------
%	PACKAGES AND OTHER DOCUMENT CONFIGURATIONS
%----------------------------------------------------------------------------------------

\documentclass[letterpaper]{twentysecondcv} % a4paper for A4

\awards {
\begin{itemize}
	\item Particiant of 2014 and 2015 Ukraine ACM ICPC.
	\item 2013: 27th All Ukrainian Olympiad in Informatics, Lugansk, Ukraine\\ \textbf{third diploma}.
\end{itemize}
}

\publications {
%\vspace{1mm}
%\\{\footnotesize{PDMU-2017 XXIX International Conference}}
\begin{itemize}
\item 2017: Constructing a unified algorithmic platform based on Voronoi diagram.
%\end{itemize}
%{\footnotesize{XV International conference "Shevchenkivska Spring"}}
%\begin{itemize}
\item 2017: Greedy approach for solving Art Gallery Problem
\end{itemize}
}

\personal {
\begin{itemize}
	\item Team player, purposeful, responsible, sociable, patient, disciplined and fast learner.
\end{itemize}
        

}
%----------------------------------------------------------------------------------------
%	 PERSONAL INFORMATION
%----------------------------------------------------------------------------------------

% If you don't need one or more of the below, just remove the content leaving the command, e.g. \cvnumberphone{}

\cvname{Vitalii Vrublevskyi} % Your name
\cvjobtitle{Software Engineer} % Job title/career
\cvlinkedin{https://www.linkedin.com/in/vitalii-vrublevskyi}% LinkedIn

\cvnumberphone{+380680550459} % Phone number
\cvmail{vitalii.vrublevskyi@gmail.com} % Email address

\cvsite{github.com/vrublevskiyvitaliy} % Personal website
\cvhaker{https://www.hackerrank.com/vrublevskyi}
\cvleetcode{https://leetcode.com/vitalii-vrublevskyi}
\cvcodeforce{http://codeforces.com/profile/Steel_Rat11}
\cvkaggle{https://www.kaggle.com/steelrat11}

%----------------------------------------------------------------------------------------

\begin{document}

\makeprofile % Print the sidebar

%----------------------------------------------------------------------------------------
%	 EDUCATION
%----------------------------------------------------------------------------------------
\section{Education}

\begin{twenty} % Environment for a list with descriptions
	\twentyitem
    	{Expected \\ June 2019}
        {Master degree in Informatics}
        {Kyiv, Ukraine}
        {Taras Shevchenko National University of Kyiv}
        {Faculty of Computer Science and Cybernetics}
        
	\twentyitem
    	{June 2017}
        {Bachelor degree with Honours in Informatics}
        {Kyiv, Ukraine}
        {Taras Shevchenko National University of Kyiv}
        {Faculty of Computer Science and Cybernetics}
        
	%\twentyitem{<dates>}{<title>}{<organization>}{<location>}{<description>}
\end{twenty}



%----------------------------------------------------------------------------------------
%	 Projects
%----------------------------------------------------------------------------------------


\section{Projects}
\begin{itemize}
	\item \projectItem
        {NBA Totalizator based on Naive Bayes.}
        {Simple predictive model of NBA game based on Naive Bayes approach using results of previous games.}
	\item \projectItem
        {Developed system for Named Entity Recognition}
        {}
	\item \projectItem
        {Implemented structured data extraction from unstructured text \giturl{https://github.com/vrublevskiyvitaliy/Law_Text_Segmentaion}}
        {}
	\item \projectItem
        {Implemented basic chess engine using Lisp \giturl{https://github.com/vrublevskiyvitaliy/ChessLisp}}
        {}
    \item \projectItem
        {Developed solver for puzzle '8' using Prolog}
        {}
    \item \projectItem
        {Designed system for automatic discrimination between printed and handwritten text in documents}
        {Used Otsu binarization, dilation and connected components to divide text into words and for each word decided class based on textural and structural features.}
    \item \projectItem
        {Created classificator for "Titanic" passengers using SVM algorithm \giturl{https://github.com/vrublevskiyvitaliy/machine-learning}}
        {}
    \item \projectItem
        {Explored scrapy python library for parsing sites \giturl{https://github.com/vrublevskiyvitaliy/scrapy_spbguru}}
        {}
    \item \projectItem
        {Implemented library for manipulation with big numbers and applied it in RSA algorithm implementation}
        {}
    \item \projectItem
        {Explored Android Camera2 API, create android app for taking photos with different focus distance \giturl{https://github.com/vrublevskiyvitaliy/Android_Camera2_API}}
        {}
    \item \projectItem
        {Explored lex \& yacc in order to parse and analyse data base description MI language \giturl{https://github.com/vrublevskiyvitaliy/LexYacc}}
        {}
    \item \projectItem
        {Implemented NRZI and MLT-3 encoding and decoding \giturl{https://github.com/vrublevskiyvitaliy/Networks}}
        {}
    \item \projectItem
        {Parallel programing}
        {Implemented parallel Dijkstra algorithm using MPI, OpenMP. Exploring CUDA for building K-d tree. I used university PARCS approach to solve knapsack problem.}
    \item \projectItem
        {Explored signal processing using least squares approach, image filtering in MatLab}
        {}
    \item \projectItem
        {Implemented minimization of deterministic finite automata in Java}
        {}
    \item \projectItem
        {Created web-based interactive system for proving predicats of first-order logic}
        {}
    \item \projectItem
        {Developed information system for Intellectual games using C\# and LINQ \giturl{https://github.com/vrublevskiyvitaliy/LINQ}}
        {}
    \item \projectItem
        {Developed aproximation of function and calculating of integral using Simpson method in Angular \giturl{https://github.com/vrublevskiyvitaliy/aproximation}}
        {}
    \item \projectItem
        {lun.ua}
        {Lun.Novostroyki - service for choosing apartments at new buildings.PHP, Python, MySQL, JS, Elasticsearch, Angular 2. Provided ideas to improve project architecture, divided tasks into stages and implemented them. }
    \item \projectItem
        {MP5 Project - WeDesign.Live}
        {Web based live collaborative platform for designing with slicer software. Developed JavaScript side of designer, architecture for constructive solid geometry (CSG) technique, implemented tree based data structure which decreased required memory and calculation time.}

\end{itemize}
\end{document} 
